\documentclass{article}
\usepackage[utf8]{inputenc}
\usepackage{graphicx}
\usepackage{hyperref}

\hypersetup{
    colorlinks=true,
    linkcolor=blue,
    filecolor=magenta,      
    urlcolor=cyan,
    pdftitle={Sharelatex Example},
    bookmarks=true,
    pdfpagemode=FullScreen,
    }

\title{User Stories}
\author{Sophie Avard}
\date{September 2019}

\begin{document}

\maketitle

\tableofcontents

\section{Introduction}
Throughout the semester I have built on the work that I put forward in the Scoping and Elaboration exercises. As I have been continuously learning new methods for organising and analysing data, my PoC has adjusted overtime. What originally started off as a project aimed at coding data (as anthropologists would say) in order to develop connections between texts, has turned into a much larger project involving data organisation, coding/tagging, and referencing. As such, I will be using tools that were not outlined in my Elaboration. That being said, I am still in the process of developing digital skills so I expect that my PoC will continue to change as I progress in this unit.\\
My PoC will focus on:
\begin{itemize}
    \item Storing multiple sources of data with relevant metadata
    \item Finding connections between sources through annotations, highlights, text searches and tags
    \item Exporting references into Overleaf 
\end{itemize}
The main tools that I will be using to complete my PoC are: Hypothes.is, OpenSemantics Desktop Search, and Zotero.\\ The tools that I will be using in order for my main tools to successfully function will be: VirtualBox, Zotfile, Zotero-Voyant-Export, and Better BibTex.

\section{User Stories}

\subsection{Text Search}
As a student, I want my data analysis tool to have a search feature so that I can easily find phrases and words across multiple sources of data.\\

As a student, I should be able to:
\begin{itemize}
    \item Open tool
    \item Open data
    \item Use search open to find specific words
\end{itemize}

\subsubsection{Quality Assurance}
For text search to be successful, I must be able to search for specific words across at least two sources.

\subsection{Tag Sources}
As a student, I want my data analysis tool to create tags and add tags to sources so that I can identify patterns.\\

As a student, I should be able to:
\begin{itemize}
    \item Start tool
    \item Select specific document 
    \item Click add new entry
    \item Enter name for tag
    \item Click save
\end{itemize}

\subsubsection{Quality Assurance}
For tagging to be successful, I must be able to create my own tags and add them to at least two sources. 

\subsection{Highlight Text}
As a student, I want my data analysis tool to allow me to highlight text to help me organise themes and connections between sources.

As a student, I should be able to:
\begin{itemize}
    \item Open tool
    \item Click file
    \item Click open file
    \item Select relevant pdf
    \item Cause tool to highlight text
    \item Save highlight 
\end{itemize}

\subsubsection{Quality Assurance} 
For highlighting to be successful I must be able to see at least two highlighted texts.

\subsection{Export Data Analysis}
As a student, I want my data analysis tool to export the tagged data in a csv format so that I can store it on my computer for later analysis. \\

As a student, I should be able to:
\begin{itemize}
    \item Open tool
    \item Open relevant file
    \item Tag the file
    \item Save the tags
    \item Cause tool to export data
    \item Save as csv format
\end{itemize}

\subsubsection{Quality Assurance}
For exporting to be successful I must be able to export in csv format.

\end{document}
